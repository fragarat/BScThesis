% ---------------------------------------------------------------------
% ---------------------------------------------------------------------
% General configuration

\usepackage{pdfpages}
\usepackage[title,titletoc]{appendix}
\usepackage{glossaries}

\usepackage{caption}
\usepackage{subcaption}

\usepackage{tikz}
\usepackage[most]{tcolorbox}

\newtcbtheorem[auto counter]{Definition}{Definition}{
	lower separated=false,
	colback=white!80!gray,
	colframe=white, fonttitle=\bfseries,
	colbacktitle=white!50!gray,
	coltitle=black,
	enhanced,
	boxed title style={colframe=black},
	attach boxed title to top left={xshift=0.5cm,yshift=-2mm},
}{def}


\newtcbtheorem[auto counter]{Theorem}{Theorem}{%
	lower separated=false,
	colback=white,
	colframe=black,fonttitle=\bfseries,
	colbacktitle=black,
	coltitle=white,
	enhanced,
	attach boxed title to top left={yshift=-0.1in,xshift=0.15in},
	boxed title style={boxrule=0pt,colframe=white,},
}{theo}

\usepackage[linesnumbered,ruled]{algorithm2e}

% Using LU fonts (Adobe Garamond Pro and Frutiger) if we compile with XeLaTeX
\usepackage{ifxetex}

\ifxetex
	\usepackage{fontspec}
	\setmainfont{AGaramondPro-Regular.otf}[
		Path = fonts/ ,
		BoldFont = AGaramondPro-Bold.otf ,
		ItalicFont = AGaramondPro-Italic.otf ,
		BoldItalicFont = AGaramondPro-BoldItalic.otf ]
	\setsansfont{FrutigerLTStd-Light.otf}[
		Path = fonts/ ,
		BoldFont = FrutigerLTStd-Roman.otf ,
		ItalicFont = FrutigerLTStd-LightItalic.otf ,
		BoldItalicFont = FrutigerLTStd-Italic.otf ]
\else
	\usepackage[latin1]{inputenc}
	\usepackage[T1]{fontenc}
	\usepackage{lmodern}
\fi

%%% BABEL %%%


\ifswedish\usepackage[swedish]{babel}\fi
\ifenglish\usepackage[english]{babel}\usepackage{csquotes}\fi
\newcommand{\sen}{\on{sen}}	

%%%%%%%%%%%%%

\usepackage[math]{blindtext}
\usepackage{xcolor}
\usepackage{array,booktabs}
\usepackage{tabularx,longtable,multicol}
\usepackage{amssymb}

%\ifenglish
%	\raggedright % Does not justify and does not divide words with hyphens
%\fi

% ---------------------------------------------------------------------
% ---------------------------------------------------------------------
% Bibliography

\usepackage[
	url = false,
	style = numeric,
	hyperref = true,
	backref = true,
	%backend = biber, % Another option is 'backend = bibtex'
	backend = bibtex,
	]{biblatex}

% ---------------------------------------------------------------------
% ---------------------------------------------------------------------
% Colored links

\usepackage{makeidx}
\makeindex


\ifAfour
	\definecolor{LUblue}{RGB}{0,0,238}
	\colorlet{linkColor}{LUblue}
\else 
	\colorlet{linkColor}{black} 
\fi

\usepackage[
	{colorlinks=true},
	{linkcolor=linkColor},
	{citecolor=linkColor},
	{urlcolor=linkColor},
	{bookmarksnumbered},
	{breaklinks},
	]{hyperref}

% ---------------------------------------------------------------------
% ---------------------------------------------------------------------
% Units according to the International System, currencies

\usepackage{eurosym}
\usepackage{siunitx}

\ifenglish
	\sisetup{output-decimal-marker={.}}
\else
	\sisetup{output-decimal-marker={,}}
\fi

\DeclareSIUnit[number-unit-product = {\;}] \EURO{\geneuro}

% ---------------------------------------------------------------------
% Custom commands and environments
% ---------------------------------------------------------------------

\usepackage{xspace}

\newcommand{\matlabr}{{\sc Matlab}$^\circledR$\xspace}
\newcommand{\simulinkr}{\textit{Simulink}$^\circledR$\xspace}
\newcommand{\matlab}{{\textsc{Matlab}}\xspace}
\newcommand{\simulink}{\textit{Simulink}\xspace}

\newcommand{\scr}{\textit{script\/}\xspace}
\newcommand{\scrs}{\textit{scripts\/}\xspace}

% ------------------------------------------------------------------------

\definecolor{mygrey}{rgb}{.925, .925, .925}



\newsavebox{\mybox}
\newenvironment{highlightedParagraph} % Highlighted paragraph
		{%
		\fboxsep = 2ex
		\fboxrule = .4pt
	  	\begin{lrbox}{\mybox}%
	  	\begin{minipage}{.85\textwidth-2\fboxsep}\itshape\parskip=2ex
		}
		{%
		\end{minipage}
	  	\end{lrbox}%
		\begin{flushright}
			\colorbox{mygrey}{\usebox{\mybox}}%
	  		%\fcolorbox{black}{mygrey}{\usebox{\mybox}}%
		\end{flushright}
		}
		
%	\newenvironment{highlightedParagraph} % If you don't like grey shading
%		{
%		\vspace{0ex}
%		\begin{quote}\noindent
%		\itshape
%		\samepage
%		\parskip=2ex
%		%\color{NavyBlue}
%		}
%		{
%		\end{quote}
%		}


% ------------------------------------------------------------------------
% Chapter summary

\newsavebox{\myboxb}
\newenvironment{Summary}
		{%
		\vspace*{-2.0cm}
		\fboxsep = 0pt
		\fboxrule = 0pt
	  	\begin{lrbox}{\myboxb}%
	  	\begin{minipage}{.85\textwidth}\itshape\parskip=2ex\parindent=2em
		}
		{%
		\end{minipage}
	  	\end{lrbox}%
		\begin{flushright}
			\usebox{\myboxb}%
		\end{flushright}
		%\vspace{0.1cm}
		}


% ---------------------------------------------------------------------
% ---------------------------------------------------------------------
% Math symbols

\newcommand{\on}{\operatorname}

% ---------------------------------------------------------------------
% ---------------------------------------------------------------------
% Theorems and examples

\ifswedish
	\newtheorem{theorem}{\upshape\bfseries Sats}[section]
	\newtheorem{lemma}{\mdseries\scshape Lemma}[section]
	\newtheorem{proposition}{\upshape\bfseries Påstående}[section]
	\newtheorem{example}{\bfseries\scshape Exempel}[section]
\fi

\ifenglish
	\newtheorem{theorem}{\upshape\bfseries Theorem}[section]
	\newtheorem{lemma}{\mdseries\scshape Lemma}[section]
	\newtheorem{proposition}{\upshape\bfseries Proposition}[section]
	\newtheorem{example}{\bfseries\scshape Example}[section]
\fi

% ---------------------------------------------------------------------
% ---------------------------------------------------------------------

\definecolor{codegreen}{rgb}{0,0.6,0}
\definecolor{codegray}{rgb}{0.5,0.5,0.5}
\definecolor{codepurple}{rgb}{0.58,0,0.82}
\definecolor{backcolour}{rgb}{0.95,0.95,0.92}
% listings
\usepackage{listings}

\lstdefinestyle{mystyle}{
    backgroundcolor=\color{backcolour},   
    commentstyle=\color{codegreen},
    keywordstyle=\color{magenta},
    numberstyle=\tiny\color{codegray},
    stringstyle=\color{codepurple},
    basicstyle=\ttfamily\footnotesize,
    breakatwhitespace=false,         
    breaklines=true,                 
    captionpos=b,                    
    keepspaces=true,                 
    numbers=left,                    
    numbersep=5pt,                  
    showspaces=false,                
    showstringspaces=false,
    showtabs=false,                  
    tabsize=2
}

\lstset{style=mystyle}