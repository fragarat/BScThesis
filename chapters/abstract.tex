% ---------------------------------------------------------------------
% ---------------------------------------------------------------------
% ---------------------------------------------------------------------

\chapter*{\abstractname}
\markboth{Abstract}{}
% ---------------------------------------------------------------------
% ---------------------------------------------------------------------
% ---------------------------------------------------------------------
Static methods based on discrete snapshots or timeslices for measuring node interaction in networks that change over time do not fully capture the dynamical nature of network evolution. The present study provides a novel continuous-time approach to network centrality, a fundamental concept in network analysis.

To address this issue, a dynamical system model driven by network's continuous-time adjacency matrix is derived from a continuous version of Katz centrality, one of the most widely used centrality measures. Using this approach, numerical experiments are conducted on various networks, synthetic and real, showing that this continuous-time framework performs better than traditional static or aggregate measures and that it is able to capture dynamic effects of communication between nodes or even changes in the network structure over time.

The most significant conclusions of this work are that the new ODE-based framework offers major improvements in accuracy and efficiency over static methods for network simulations. By using advanced numerical ODE solvers, time discretization is performed automatically "under the hood" in an optimal and efficient manner, allowing the system to adapt to sudden and significant changes in network behavior. The ODE system’s property of downweighting information over time also enables real-time monitoring of centrality rankings without the need to store or account for all previous node interaction history. Moreover, it is shown why tracking good receivers of information in a dynamic network is cheaper than tracking good broadcasters from a computational cost perspective. 

Overall, this dynamical systems approach to network centrality can lead to new insights into the behavior of complex networks and has implications for a wide range of applications, from social networks to fluid mechanics, where network dynamics play a crucial role.


% ---------------------------------------------------------------------
