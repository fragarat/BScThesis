% ----------------------------------------------------------------------------------
% ----------------------------------------------------------------------------------

\chapter{Further studies}
\label{chap:further}
 
\section*{Further generalization of the framework on dynamical systems at lower dimensions}
A natural continuation of this study could be applied to different types of dynamical systems where only a few nodes compared to the total number of nodes are really significant in the behavior and evolution of such systems \cite{grindrod2014dynamical}. For instance, consider an evolving network $\mathbf{A}(t)$ with $N \times N$ dimensions, where $N$ is very large. If a smaller subset of $M\ll N$ nodes is identified as significant, it might be worthwhile to explore an ordinary differential equation involving $V(t)\in \mathbb{R}^{M\times M}$ with $M\times M$ dimensions. The equation could be in the form of 

$$V'(t)=P(V(t)) + Q(V(t))F(A(t)),$$ where $P$ and $Q$ are polynomial or matrix-valued functions, and $F: [0, 1]^{N×N} \to \mathbb{R}^{M\times M}$ is an appropriate matrix-valued mapping. This kind of system would allow us to reduce the interactions among all $N$ nodes to the subset of interest, and then measure the resulting changes in behavior in this lower dimension.

\section*{Potential use of the framework in applied scientific fields} 
Further studies could explore the potential of centrality measures such as the generalized Katz measure derived from this study and dynamic network broadcast/receive analysis in applied fields like fluid dynamics, \chreplaced{atmospheric}{atmosferic} science or engineering: 

\begin{itemize}
  \item These measures could be used to study the complex spatiotemporal dynamics of turbulent flows, and identify key locations or structures that could be targeted for control or optimization. 
  \item They can also be a useful tool in the study of fluid combustion, by identifying the critical points in the combustion system where the combustion reaction is most likely to be affected by various factors such as temperature, pressure, and turbulence. This information can be used later to optimize the combustion process and improve its efficiency.
\end{itemize}
	
Many other scientific areas are suitable to the application of this novel framework such as social networks, traffic flow, transportation or power grids, neural networks \chreplaced{etc.}{...} since a huge number of real problems can be modeled through the use of evolving networks.

Overall, the application of centrality measures has the potential to provide valuable insights into the behavior of complex systems in multiple and diverse fields, enabling the development of more effective control and optimization strategies.

\chcomment{For the references: When citing a book, include the address (i.e. city) of the publisher.}